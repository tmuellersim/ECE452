\documentclass[10pt, oneside, letter]{article}

\usepackage{amsmath, mathtools, amsfonts, mathrsfs, enumitem, listings, color, graphicx, caption, float, indentfirst, titlesec, anysize, hyperref, fancyhdr, textcomp, lipsum, parskip, titling}

%\usepackage[showframe]{geometry}
%\usepackage{layout}

\setlength{\droptitle}{-10ex}

\pagestyle{fancy}
\lhead{Tim Mueller-Sim}
\rhead{University of Illinois at Chicago}
\chead{ECE452 HW\#1}
\renewcommand{\headrulewidth}{0.4pt}

\begin{document}

\title{{\sc ECE 452} - {\sc Robotics: Algorithms and Control} \\ {\large \sc Homework \#1}}
\author{Tim \textsc{Mueller-Sim} \\ {\tt tmuelle2@uic.edu} \\ \textsc{659991541} }
\date{January 29, 2015}

\vspace{-5ex}

\maketitle 

\vspace{-5ex}
\noindent \hrulefill

	\begin{enumerate}

	%Problem 1
	\item Given the vectors below, compute the following:
	$$\vec{v}_1 = 
	\begin{bmatrix*}[r]
	2 \\ -3 \\ 1
	\end{bmatrix*}
	\quad
	\vec{v}_2 = 
	\begin{bmatrix*}[r]
	2 \\ -1 \\ 1 \\ -3
	\end{bmatrix*}
	\quad
	\vec{v}_3 = 
	\begin{bmatrix*}[r]
	1 \\ -5 \\ -2
	\end{bmatrix*}
	\quad
	\vec{v}_4 = 
	\begin{bmatrix*}[r]
	1 \\ -2 \\ 3 \\ -3
	\end{bmatrix*}
	$$

		\begin{enumerate}
		% 1a
		\item $\vec{v}_1 + \vec{v}_2: \quad$
		Cannot compute the sum of the two vectors as they are not the same size.
	
		% 1b
		\item $\vec{v}_1 - \vec{v}_3 = \begin{bmatrix*}[r]
		1 \\ 2 \\ 3
		\end{bmatrix*}$
	
		% 1c
		\item $\vec{v}_2 \cdot \vec{v}_3: \quad$
		Cannot compute the inner product of the two vectors, as they are not the same size.
		
		% 1d
		\item $\vec{v}_2 \cdot \vec{v}_4 = 16$
		
		%1e
		\item {$\vec{v}_1 \times \vec{v}_3 = 
		\begin{bmatrix*}[r]
		11 \\ 5 \\ -7
		\end{bmatrix*}$}
		
		\end{enumerate}

	% Problem 2
	\item Given the matrices below, compute the following:
	
	$$ A = 
	\begin{bmatrix*}[r]
	-4 & 5 & 2 \\ -5 & 0 & -5 \\ 5 & 0 & 3
	\end{bmatrix*}	
	\quad
	B = 
	\begin{bmatrix*}[r]
	2 & 1 & 3 \\ -3 & 0 & -5 \\ -3 & -1 & 0 \\ -2 & 5 & -2
	\end{bmatrix*}	
	\quad
	C = 
	\begin{bmatrix*}[r]
	-1 & 2 & 1 \\ 1 & -2 & -3 \\ 0 & 1 & -2
	\end{bmatrix*}	
	$$		
	
		\begin{enumerate}
		% 2a
		\item $A \cdot B: \quad$
			Cannot perform matrix multiplication, as the column dimension of A does not equal the row dimension of B.
			\begin{center}
			$A \in \mathbb{R}^{3 \times 3} \quad \mbox{and} \quad B \in \mathbb{R}^{4 \times 3}$
			\end{center}
			
		% 2b
		\item $B \cdot A = 
			\begin{bmatrix*}[r]
			2 & 10 & 8 \\ -13 & -15 & -21 \\ 17 & -15 & -1 \\ -27 & -10 & -35
			\end{bmatrix*}
			$
		
		%2c
		\item $A + B: \quad$
		Cannot add the two matrices, as their dimensions do not match.
		\begin{center}
		$A \in \mathbb{R}^{3 \times 3} \quad \mbox{and} \quad B \in \mathbb{R}^{4 \times 3}$
		\end{center}
		
		% 2d
		\item $A - C = 
		\begin{bmatrix*}[r]
		-3 & 3 & 1 \\ -6 & 2 & -2 \\ 5 & -1 & 5
		\end{bmatrix*}$
		
		% 2e
		\item $A \cdot C = 
		\begin{bmatrix*}[r]
		9 & -16 & -23 \\ 5 & -15 & 5 \\ -5 & 13 & -1
		\end{bmatrix*}$
		
		% 2f
		\item $C \cdot A = 
		\begin{bmatrix*}[r]
		-1 & -5 & -9 \\ -9 & 5 & 3 \\ -15 & 0 & -11
		\end{bmatrix*}$
		
		% 2g
		\item Cannot multiply matrix with itself, as it is not a square matrix.
		$$B \in \mathbb{R}^{4 \times 3}$$
		
		\end{enumerate}
	

	% Problem 3
	
	\item Compute the eigenvalues and eigenvectors of the following matrix:
	
	$$A = \begin{bmatrix*}[r]
	-3 & -1 & -2 \\ 4 & 2 & 2 \\ 6 & 2 & 4
	\end{bmatrix*}$$
	
	Using the equation $\mbox{det}(A - \lambda \cdot I) = 0$, we can solve for the eigenvalues $\lambda$:
	
	$$\mbox{det}(A - \lambda \cdot I) = \mbox{det}\begin{bmatrix*}[r]
	-\lambda-3 & -1 & -2 \\ 4 & 2-\lambda & 2 \\ 6 & 2 & 4-\lambda
	\end{bmatrix*} = -\lambda^3 + 3\lambda^2 - 2\lambda = 0
	$$
	
	\begin{center}
	\framebox[65]{$\lambda = 0, 1, 2$}
	\end{center}
	
	Using the equations $A\nu = \lambda\nu$ and $(A - \lambda I)\cdot \nu = 0$ we can compute the eigenvectors $\nu$:
	
		\quad \bf Case 1: $\lambda = 0$ \rm 
		
		$$(A + 0 \cdot I)\nu = 0 = A \nu = \left\{ \begin{array}{r} -3x - y - 2z \\ 4x + 2y + 2z \\ 6x + 2y + 4z \end{array} \right.$$ 
		
		\quad Solving for the system equations, we obtain 
		
		$$\nu =\left( \begin{array}{r} x \\ y \\ z \end{array} \right) = c\begin{bmatrix*}[r] 1 \\ -1 \\ -1 
		\end{bmatrix*}$$
		
		\quad where c is an arbitrary constant.
		
		\quad \bf Case 2: $\lambda = 1$ \rm 
		Solving similarly to Case 1 above, we obtain
		
		$$\nu = c\begin{bmatrix*}[r] 1 \\ 0 \\ -2 \end{bmatrix*}$$ 
		
		\quad \bf Case 3: $\lambda = 2$ \rm
		Solving similarly to Case 1 above, we obtain
		
		$$\nu = c\begin{bmatrix*}[r] 1 \\ -1 \\ -2 \end{bmatrix*}$$ 
		
	
	% Problem 4
	\item Using the Laplace transform formula, compute $e^{At}$ where 
	
	$$A = \begin{bmatrix*}[r] -2 & 1 \\ -1 & 0 \end{bmatrix*}$$ 
	
	
	As outlined in class, $e^{At} = \mathcal{L}^{-1}[(sI - A)^{-1}]$. \\
	
	$$(sI - A)^{-1} = 
	\begin{bmatrix*}[c] s + 2 & -1 \\ 1 & s
	\end{bmatrix*}^{-1} = 
	\arraycolsep=2pt\def\arraystretch{2.2}
	\begin{bmatrix*}[c] \dfrac{s}{s^2 + 2s + 1} & \dfrac{1}{s^2 + 2s + 1} \\ -\dfrac{1}{s^2 + 2s + 1} & \dfrac{s + 2}{s^2 + 2s + 1}
	\end{bmatrix*} $$ 
	
	$$\mathcal{L}^{-1}
	\arraycolsep=2pt\def\arraystretch{2.2}
	\begin{bmatrix*}[c] \dfrac{s}{s^2 + 2s + 1} & \dfrac{1}{s^2 + 2s + 1} \\ -\dfrac{1}{s^2 + 2s + 1} & \dfrac{s + 2}{s^2 + 2s + 1}
	\end{bmatrix*} = \begin{bmatrix*}[c]
	e^{-t} -te^{-t} & te^{-t} \\ -te^{-t} & e^{-t} + te^{-t}
	\end{bmatrix*} = e^{At}$$ \\
	 
	Calculations were verified in MATLAB using the \tt ilaplace \rm function \\
	
	% Problem 5
	\item Compute $e^{At}$ for $t = 3$ where 

	$$A = \begin{bmatrix*}[r] -6 & 5 \\ -4 & 3
	\end{bmatrix*}$$ 
	
	Solved using the \tt expm \rm function in MATLAB:
	
	$$At = \begin{bmatrix*}[r] -18 & 15 \\ -12 & 9
	\end{bmatrix*}$$
	\vspace{2ex}
	$$e^{At} = \begin{bmatrix*}[r] -0.1868 & 0.2365 \\ -0.1892 & 0.2390
	\end{bmatrix*}$$ 
	
	
	\end{enumerate}


\end{document}   